
\documentclass[12pt, letterpaper]{article}
\usepackage[utf8]{inputenc}
\usepackage{pgfplots}
\usepackage{subcaption}
\usepackage[english,main=russian]{babel}
\usepackage[T1]{fontenc}
\usepackage[style=numeric,sorting=none]{biblatex}
\addbibresource{refs.bib}
\nocite{*}

\title{Определение степени усталости человека по фотографии}
\date{\today}
\author{Муртазин Виталий, Фасхутдинов Тимур, \and
Тимергалин Денис, Саяхов Ильфат, \and
Ямиков Рустем, Байрамов Азат, Туманов Никита
}

\begin{document}

    \maketitle

    \begin{abstract}
        В этой статье будут описаны решения проблемы распознавания усталости человека по изображению его лица.
        В ходе данной работы сначала будут рассмотрены уже существующие алгоритмы.
        Затем будет предложена и реализована модель собственной архитектуры.
        В заключении будет проведено сравнение качества результатов представленного нами решения с иными рассмотренными вариантами.
    \end{abstract}

    \section{Вступление}\label{sec:introduction}
    Психическая усталость представляет собой психобиологическое состояние, вызванное длительными периодами вынужденной умственной активности, и имеет последствия для многих аспектов повседневной жизни.\cite{hullermeier_fuzzy_2009}
    Было обнаружено, что на рабочем месте умственная усталость приводит к повышению риска ошибки.

    Такие ошибки могут привести к снижению производительности на рабочем месте, а также трагическим последствиям.

    Одна из отраслей, в которой влияние усталости на безопасность уже давно признано - это транспорт.
    Национальное управление безопасности дорожного движения при усталости оценивает, что по крайней мере 100 000 аварий, о которых сообщает полиция, ежегодно происходят из-за усталости водителя;
    в результате этих аварий погибло около 1550 человек, было ранено 71000 человек, а денежные убытки составили 12,5 млрд долларов США.\cite{lancichinetti_benchmarks_2009}

    Только за 2017 год на территории Российской Федерации зарегистрировано порядка 126 тысяч дорожно-транспортных происшествий.
    В среднем каждое пятое ДТП происходит по вине водителей, заснувших или испытывающих сонливость за рулем, утомленность водителей является причиной 25\% всех ДТП со смертельным исходом.

    Исходя из вышеперечисленной статистики можно сделать вывод, что проблема усталости человека на рабочем месте актуальна на данный момент и требует решения.
    Результат исследования позволит снизить риск несчастных случаев по причине утомления работника.

    Существует множество признаков утомления, некоторые из которых можно обнаружить с помощью камеры.

    В последние годы обработка изображений лиц людей используется во многих приложениях, таких как распознавание лиц, обнаружение глаз при анализе лиц, отслеживание взгляда и т.д.
    Среди всех этих исследований первым шагом обычно является определение местоположения лица.
    В последнее время методы распознавания лиц постепенно развиваются.
    При наблюдении за изображением лица наиболее заметными чертами лица обычно являются места с очевидными краями, особенно контуры глаз.

    Целью данного исследования является создание программного продукта, определяющего усталость человека по фотографии с использованием сверточных нейронных сетей.
    Объектом исследования является детектирование признаков усталости (определение степени усталости) человека на рабочем месте.

    \section{Проблема}\label{sec:problem}

    В современном мире актуальна проблема повышенной утомляемости людей из-за чрезмерной нагрузки.
    А чем выше усталость человека, тем выше вероятность ошибки.
    В некоторых сферах цена ошибки очень высока.
    Она может привести к убыткам, а в худшем случае может нанести ущерб здоровью.

    Вот примеры из нескольких статей:

    К личностным (психофизиологическим) причинам производственного травматизма условно можно отнести физические и нервно-психические перегрузки работника, приводящие к его ошибочным действиям.
    Человек может совершать ошибочные действия из-за утомления, вызванного большими физическими (статическими и динамическими) перегрузками, умственным перенапряжением, перенапряжением анализаторов (зрительного, слухового, тактильного), монотонностью труда, стрессовыми ситуациями, болезненным состоянием.

    На внимание работника влияют несколько факторов, в том числе усталость, нагрузка на работе и стресс в рабочей среде.
    В небезопасной рабочей среде и при комбинации названных факторов вопрос не в том, произойдет ли на работе несчастный случай, а в том, как скоро он произойдет", – обращает внимание Л. Матисане.

    Экспериментально доказано, что все аварии и несчастные случаи тесно связаны с наступлением усталости.
    А средством противодействия развития усталости выступает фактор заинтересованности в работе.
    Поэтому можно утверждать, что на склонность к несчастным случаям влияет также уровень заинтересованности и удовлетворенности человека своей работой.

    Усталость ведет к потере внимания и концентрации.
    Кроме того, уставший человек невольно пытается закончить порученную ему работу как можно быстрее, что в умножении на нашу привычку работать в режиме аврала, которую мы подробно разобрали в Главе «Исторические особенности развития трудовых отношений в России» приводит к плачевным последствиям.

    Но, к сожалению, рабочие места не всегда соответствуют требования законодательства.
    Кроме того, работник может получить травму в результате причин субъективного характера.
    Например, усталость, головокружение, невнимательность, нестабильный эмоциональный фон — все эти причины могут быть источником производственной травмы.

    Однако если вовремя определять усталость людей и не допускать их до рабочего процесса, то можно минимизировать последствия ошибок.

    \begin{refsection}[refs.bib]
        \nocite{*}
        \printbibliography
    \end{refsection}

\end{document}